


I express my deepest gratitude to my supervisor, Prof. Dr. Mário Ramirez, for his mentorship and support throughout the years. Our invaluable discussions about the subjects presented in this thesis were crucial to reaching this point and for me to be able to implement many of the concepts that we discussed. His eagle-eyed view of the concepts and results revealed limitations in the approaches I was trying to follow, steering me towards more fruitful endeavors.

To all of the colleagues in the MRamirez lab, thank you for all the shared moments and knowledge about streptococci and microbiology in general. My work was heavily focused on the technical aspects of software development, which could make me lose sight of the approaches that are more biologically relevant. Working in an environment where people have strong knowledge about the biology of bacterial pathogens and a close link to epidemiological and clinically relevant applications helped contextualize my work and guide my efforts more effectively.

Special thanks to Inês, Pedro, and Joana for their friendship and for keeping things a little more lively and interesting. It takes a lot of energy to deal with an introvert, but you had enough patience. When I arrived at the lab, I was only a "proto-binfie". I improved in great part due to the mentorship of Inês and through the knowledge shared with her and Pedro while working on several projects. That period allowed me to learn and refine multiple skills and greatly contributed to the type of "binfie" that I have become.

To all friends and family who supported me during this journey, thank you for being present, caring, and sharing. Every casual moment, every conversation, even if fleeting, are important aspects of what makes life worthwhile, .




And finally, to those who will ever be,

