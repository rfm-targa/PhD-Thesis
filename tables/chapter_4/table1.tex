\begin{table}[!ht]
    \centering
    \resizebox{0.8\linewidth}{!}{
    \begin{threeparttable}[b]
    \caption[Simpson’s index of diversity and 95\% confidence intervals for the typing methods used to characterize 265 \textit{S. pyogenes} isolates recovered in Portugal.]{Simpson’s index of diversity and 95\% confidence intervals for the typing methods used to characterize 265 \textit{S. pyogenes} isolates recovered in Portugal.}
    \label{tab:ch4_table1}
    \begin{tabular}{@{}lll@{}}
        \toprule
        \multicolumn{1}{|c|}{Typing method} & \multicolumn{1}{|c|}{No. of partitions} & \multicolumn{1}{|c|}{SID (CI$_{95\%})$\tnote{c}} \\ \midrule
        \textit{emm} type & 4 & 0.742 (0.727-0.756) \\
        ST & 15 & 0.826 (0.800-0.852) \\
        T-type\tnote{a} & 6 & 0.744 (0.720-0.768) \\
        SAg profile & 19 & 0.835 (0.813-0.857) \\
        PFGE & 16 & 0.792 (0.766-0.817) \\
        MST$_{1000}$\tnote{b} & 5 & 0.743 (0.728-0.758) \\
        MST$_{45}$\tnote{b} & 15 & 0.807 (0.779-0.835) \\
        cgMLST-100 & 245 & 0.999 (0.998-1.000) \\
        \bottomrule
    \end{tabular}
    \begin{tablenotes}
       \item [a] {\footnotesize The SID for T type was calculated for the subset of 248 isolates with a defined T type (17 isolates were nontypeable).}
       \item [b] {\footnotesize Groups of isolates linked by up to \textit{n} different loci in the \ac{MST} (MST$_n$).}
       \item [c] {\footnotesize SID, Simpson’s index of diversity; CI$_{95\%}$, 95\% confidence interval.}
    \end{tablenotes}
    \end{threeparttable}
    }
\end{table}