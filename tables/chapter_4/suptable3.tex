\begin{table}[h!]
    \centering
    \resizebox{0.95\linewidth}{!}{
    \begin{threeparttable}[b]
    \caption{Adjusted Rand (\ac{AR}) values for the clustering methods used in the analysis of 265 \textit{S. pyogenes} isolates recovered in Portugal [Dataset 1 \cite{friaes_supplemental_2023}].}
    \label{tab:ch4_tableS3}
    \begin{tabular}{@{}lllllll@{}}
        \toprule
        \multicolumn{1}{|c|}{} & \multicolumn{1}{|c|}{SAg profile} & \multicolumn{1}{|c|}{\textit{emm} type} & \multicolumn{1}{|c|}{ST} & \multicolumn{1}{|c|}{PFGE} & \multicolumn{1}{|c|}{T-type\tnote{a}} & \multicolumn{1}{|c|}{$MST_{1000}$\tnote{b}} \\ \midrule
        \textit{emm} type & 0.725 & ~ & ~ & ~ & ~ & ~ \\
        ST & 0.65 & 0.755 & ~ & ~ & ~ & ~ \\
        PFGE & 0.705 & 0.861 & 0.675 & ~ & ~ & ~ \\
        T-type\tnote{a} & 0.649 & 0.927 & 0.755 & 0.806 & ~ & ~ \\
        $MST_{1000}$\tnote{b} & 0.729 & 0.996 & 0.759 & 0.865 & 0.927 & ~ \\
        $MST_{45}$\tnote{b} & 0.729 & 0.815 & 0.709 & 0.846 & 0.744 & 0.818 \\
        \bottomrule
    \end{tabular}
    \begin{tablenotes}
       \item [a] {\footnotesize The \ac{AR} values for T-type were calculated for the subset of 248 isolates with a defined T-type (17 isolates were non-typeable).}
       \item [b] {\footnotesize Groups of isolates linked by up to \textit{n} different loci in the MST (MST\textit{n}).}
    \end{tablenotes}
    \end{threeparttable}
    }
\end{table}
