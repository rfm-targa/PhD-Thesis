


As infecções bacterianas têm um grande impacto nos sistemas de saúde pública ao nível global, causando perdas económicas e de saúde significativas. A emergência de tecnologias de sequenciação de DNA de alto débito precisas e com custo reduzido, especialmente por permitirem a sequenciação do genoma completo de genomas bacterianos, revolucionou a caracterização de espécies bacterianas em aplicações como a vigilância e investigação de surtos. Estas tecnologias têm sido adoptadas por instituições de investigação e de saúde pública, levando ao aumento da resolução em diversas aplicações que complementam ou substituem técnicas clássicas de fenotipagem e moleculares. A escala e complexidade dos dados gerados requer mais capacidade de armazenamento de dados e melhores métodos computacionais de análise. Estes requisitos potenciaram um crescimento tremendo na área de bioinformática, que se tornou numa das partes integrantes das abordagens ómicas como a genómica. Na genómica de bactérias, em particular, os métodos bioinformática tornaram-se essenciais para caracterizar estirpes bacterianas, melhorando a resolução de aplicações em áreas como a vigilância de doenças infecciosas e investigação de surtos, bem como para o estudo da estrutura e evolução populacional. A sequenciação completa do genoma torna possível uma abordagem baseada apenas na composição das sequências para a caracterização bacteriana, permitindo abordagens mais dirigidas para identificar genes de interesse ou a análise da diversidade genética completa. O aumento do número de genomas bacterianos completos ou parciais em bases de dados públicas permitiu aos investigadores estudar a estrutura e a variabilidade dos genomas bacterianos e encorajou o desenvolvimento de métodos de tipagem de alta resolução, como os métodos gene-by-gene e baseados em SNPs. Apesar de ambas as abordagens serem frequentemente aplicadas em genómica comparativa, quer separadamente ou em combinação, para estudar a diversidade de populações bacterianas, os métodos gene-by-gene como whole- e core-genome multilocus sequence typing (wg/cgMLST) têm sido adoptados com maior frequência por instituições de investigação e saúde pública. Os métodos wg/cgMLST permitem criar esquemas que capturam a diversidade genética de espécies de interesse. Estes esquemas são utilizados para a caracterização de estirpes bacterianas e podem ser atualizados com novos alelos ao longo de tempo para maximizar a sua aplicabilidade a longo prazo. Este é um processo gradual que é geralmente efectuado por plataformas web que centralizam as análises. A eficiência das análises em sistemas centralizados pode ser suficiente para vigilância de rotina, mas levanta questões de escalabilidade quando é necessário efectuar análises de grande escala em tempo reduzido, especialmente à medida que o número de genomas disponíveis aumenta. Para além disso, os sistemas centralizados requerem que os utilizadores disponibilizem os seus dados, o que poderá não ser possível para utilizadores que trabalhem sob políticas de partilha de dados mais restringentes. A maior parte das análises também são efetuadas ao nível do genoma core, focando-se no conjunto de loci que estão presentes na maior parte das estirpes, mas descartando loci menos frequentes que constituem uma parte muito significativa da diversidade genética e que podem ser determinantes para características fenotípicas relevantes, como virulência e resistência aos antimicrobianos. Deste modo, o desenvolvimento de métodos que permitam análises de grande escala locais, e que também sejam capazes de integrar a diversidade de loci do genoma acessório de forma mais precisa, pode minimizar questões relativas a escalabilidade e privacidade dos dados e expandir consideravelmente a resolução dos métodos wg/cgMLST. A presente tese apresenta métodos que procuram melhorar a escalabilidade, exactidão, e a interoperabilidade de métodos wg/cgMLST.

A ferramenta chewBBACA para wg/cgMLST serviu de base para explorar e implementar novos métodos para melhorar análises wg/cgMLST. Os novos métodos foram implementados na ferramenta chewBBACA 3, que corresponde a uma reimplementação completa do seu precursor, chewBBACA 2. Comparativamente ao chewBBACA 2, que avaliava as sequências codificantes previstas para cada genoma separadamente, o chewBBACA 3 identifica e armazena a lista de sequências codificantes distintas previstas a partir de todos os genomas para possibilitar correspondências exactas rápidas e não redundantes e a classificação ao nível do DNA e da proteína baseada em comparações de hashes de sequências. O chewBBACA 3 também complementa a identificação de alelos com base em alinhamento com métodos alignment-free, mais especificamente, clustering com base em minimizers, permitindo uma identificação mais rápida e exacta de alelos. A criação de esquemas com o chewBBACA 3 é até 55 vezes mais raṕida do que com o chewBBACA 2 e identificar até 10\% mais loci, permitindo capturar mais da diversidade de espécies bacterianas. A identificação de alelos com o chewBBACA 3 é 20.3 a 51.9 vezes mais rápida do que com o chewBBACA 2 e outro método comparável, respectivamente. Para além disso, o chewBBACA 3 classifica mais sequências codificantes e escala melhor do que os outros métodos, permitindo análises wg/cgMLST a grande escala em tempo reduzido e com recursos disponíveis num portátil. O chewBBACA 3 inclui funcionalidades para criar reports interactivos para uma análise intuitiva e detalhada dos esquemas e resultados de wg/cgMLST. Os resultados e funcionalidades dos reports permitem explorar a diversidade genética e identificar grupos de estirpes semelhantes, aspectos relevantes para vigilância, detecção de surtos, e estudos populacionais.

Um Web service, chamado Chewie-NS, foi implementado para disponibilizar esquemas wg/cgMLST e possibilitar análises locais e privadas com base numa nomenclatura alélica comum. O Chewie-NS combina várias tecnologias através de containerization, sendo constituído por dois componentes: um componente backend e outro de frontend. O componente de backend inclui as bases de dados para armazenar e gerir os dados dos utilizadores e dos esquemas wg/cgMLST, bem como uma API que aceita e processa os pedidos dos utilizadores e disponibiliza a data para o componente frontend. A API permitiu desenvolver um conjunto de módulos para integração com o chewBBACA 3 de forma a disponibilizar funcionalidades para descarregar, carregar e sincronizar esquemas. A integração com o chewBBACA 3 permite aos utilizadores preparar depressa um esquema wg/cgMLST para análises locais e escaláveis, descarregar novos alelos adicionados aos esquemas remotos no Chewie-NS, e contribuir com novos alelos identificados localmente apenas se desejarem. O processo de sincronização mantém as nomenclaturas alélicas utilizadas pelos esquemas locais e remotos sincronizadas para assegurar que os resultados são comparáveis. Esta abordagem descentralizada contrasta com o modelo centralizado adoptado por outras plataformas web para wg/cgMLST, que requerem que os utilizadores carreguem os seus dados para a plataforma, levantando questões de escalabilidade e confidencialidade de dados. O componente frontend trata do render do website do Chewie-NS, disponibilizando a lista de esquemas disponíveis para descarregar e estatísticas relevantes sobre a composição dos esquemas e a diversidade genética. O website também também disponibiliza uma ligação para uma interface gráfica da API, permitindo que utilizadores com diferentes níveis de proficiência explorem a API de forma mais intuitiva e tenham acesso a dados detalhados sobre os esquemas e loci.

Um novo esquema wgMLST para Streptococcus pyogenes, constituído por 3,044 loci, foi desenvolvido a partir de datasets representativos da diversidade da espécie. Os loci do esquema foram anotados funcionalmente com base em várias fontes. Um processo de curadoria por parte de uma especialista na espécie permitiu validar as anotações e refinar o esquema através da substituição ou remoção de loci espúrios. As soluções criadas para resolver os problemas identificados durante o processo de refinamento podem ser integradas em workflows para melhorar a qualidade de esquemas e análises wgMLST. O esquema wgMLST anotado melhora a resolução das análises comparativamente com métodos de tipagem mais clássicos, como PFGE e MLST de sete genes, apresentando desempenho semelhante a métodos baseados em SNPs. A utilização de um esquema wgMLST, em vez de um esquema cgMLST mais estrito, permite análises cgMLST escaláveis em que o conjunto de core loci é ajustado com base no dataset em análise. O esquema proporciona alto poder discriminatório para caracterizar e distinguir estirpes num dataset representante da diversidade global de S. pyogenes, bem como num contexto de surto para distinguir estirpes de linhagens recentes.

Em conclusão, os métodos e resultados apresentados nesta tese procuram melhorar as actuais abordagens para a caracterização de bactérias com base em sequenciação completa do genoma. O chewBBACA 3 reduz os requisitos para análises wg/cgMLST escaláveis e detalhadas. O Chewie-NS, mesmo que não seja tão completo do ponto de vista de funcionalidades comparativamente com outras plataformas bem estabelecidas, simplifica o acesso a esquemas e procura minimizar problemas relacionados com escalabilidade e confidencialidade dos dados. O esquema wgMLST para S. pyogenes permite uma caracterização detalhada de estirpes a qualquer nível de resolução, e o seu desenvolvimento permitiu identificar problemas comuns e soluções para melhorar a qualidade de esquemas wg/cgMLST.


