As infecções bacterianas têm um grande impacto nos sistemas de saúde pública a um nível global, causando perdas significativas tanto do ponto de vista económico como de saúde. A emergência de tecnologias de sequenciação de ADN de alto débito de elevada precisão e com custo reduzido, especialmente por permitirem a sequenciação total do genoma bacteriano, revolucionou a caracterização de espécies bacterianas em aplicações como a vigilância e investigação de surtos. Estas tecnologias têm sido adoptadas por instituições de investigação e de saúde pública, contribuindo para o aumento da resolução das análises em diversas aplicações e complementando ou substituindo técnicas clássicas de fenotipagem e moleculares. A escala e complexidade dos dados gerados requer maior capacidade de armazenamento de dados e melhores métodos computacionais para análise de dados. Estes requisitos potenciaram um crescimento tremendo na área de bioinformática, que se tornou numa das partes integrantes das abordagens ómicas como a genómica. Na genómica de bactérias, em particular, os métodos bioinformática tornaram-se essenciais para caracterizar estirpes bacterianas, aumentando a resolução de aplicações em áreas como a vigilância de doenças infecciosas e investigação de surtos, bem como para o estudo da estrutura e evolução das populações bacterianas. A sequenciação total do genoma torna possível uma abordagem baseada apenas na composição das sequências para a caracterização bacteriana, permitindo abordagens mais dirigidas para identificar genes de interesse ou a análise da diversidade genética completa. O aumento do número de genomas bacterianos completos ou parciais disponíveis em bases de dados públicas permitiu aos investigadores estudar a estrutura e a variabilidade dos genomas bacterianos e encorajou o desenvolvimento de métodos de tipagem de alta resolução, como os métodos \textit{gene-by-gene} e baseados em \textit{SNPs}. Apesar de ambas as abordagens serem frequentemente aplicadas em genómica comparativa, quer separadamente ou em combinação, para estudar a diversidade de populações bacterianas, os métodos \textit{gene-by-gene} como \textit{whole-} e \textit{core-genome multilocus sequence typing} (\textit{wg/cgMLST}) têm sido adoptados com maior frequência por instituições de investigação e saúde pública. Os métodos \textit{wg/cgMLST} permitem criar esquemas que capturam a diversidade genética de espécies de interesse. Estes esquemas são utilizados para a caracterização de estirpes bacterianas e podem ser actualizados com novos alelos ao longo do tempo para maximizar a sua aplicabilidade a longo prazo. Este é um processo gradual que é geralmente efectuado por plataformas \textit{web} que centralizam as análises. A eficiência das análises em sistemas centralizados pode ser suficiente para actividades de vigilância, mas levanta questões de escalabilidade quando é necessário efectuar análises de grande escala em tempo reduzido, especialmente à medida que o número de genomas disponíveis aumenta. Para além disso, os sistemas centralizados requerem que os utilizadores disponibilizem os seus dados, o que poderá não ser possível para utilizadores que trabalhem sob políticas de partilha de dados mais restringentes. A maior parte das análises também são efetuadas ao nível do genoma \textit{core}, focando-se no conjunto de \textit{loci} que estão presentes na maior parte das estirpes, mas descartando \textit{loci} menos frequentes que constituem uma parte muito significativa da diversidade genética e que podem ser determinantes para características fenotípicas relevantes, como virulência e resistência aos antimicrobianos. Deste modo, o desenvolvimento de métodos que permitam análises de grande escala localmente, e que também sejam capazes de integrar a diversidade de \textit{loci} do genoma acessório de forma mais precisa, pode minimizar questões relativas a escalabilidade e confidencialidade dos dados e expandir consideravelmente a resolução dos métodos \textit{wg/cgMLST}. A presente tese apresenta métodos que procuram melhorar a escalabilidade, exactidão, e a interoperabilidade de métodos \textit{wg/cgMLST}.

A ferramenta chewBBACA, utilizada para nálises \textit{wg/cgMLST}, serviu de base para explorar e implementar novos métodos para melhorar os métodos \textit{wg/cgMLST}. Os novos métodos foram implementados na ferramenta chewBBACA 3, que corresponde a uma reimplementação completa do seu precursor, o chewBBACA 2. Comparativamente ao chewBBACA 2, que avaliava as sequências codificantes previstas para cada genoma separadamente, o chewBBACA 3 identifica e armazena a lista de sequências codificantes distintas previstas a partir de todos os genomas para possibilitar correspondências exactas rápidas e não redundantes e a classificação ao nível do ADN e da proteína com base em comparações de \textit{hashes} de sequências. O chewBBACA 3 também complementa a identificação de alelos com base em alinhamento com métodos \textit{alignment-free}, mais especificamente, \textit{clustering} com base em \textit{minimizers}, permitindo uma identificação mais rápida e precisa de alelos. A criação de esquemas com o chewBBACA 3 é até 55 vezes mais rápida do que com o chewBBACA 2 e identifica até 10\% mais \textit{loci}, permitindo capturar mais da diversidade de espécies bacterianas. A identificação de alelos com o chewBBACA 3 é 20.3 a 51.9 vezes mais rápida do que com o chewBBACA 2 e outro método comparável, respectivamente. Para além disso, o chewBBACA 3 classifica mais sequências codificantes e escala melhor do que os outros métodos, permitindo análises \textit{wg/cgMLST} a grande escala em tempo reduzido e com recursos disponíveis num portátil. O chewBBACA 3 inclui funcionalidades para criar relatórios interactivos para uma análise intuitiva e detalhada dos esquemas e resultados \textit{wg/cgMLST}. Os resultados e funcionalidades dos relatórios permitem explorar a diversidade genética e identificar grupos de estirpes semelhantes, aspectos relevantes para vigilância, detecção de surtos, e estudos populacionais.

Um \textit{Web service}, chamado Chewie-NS, foi implementado para disponibilizar esquemas \textit{wg/cgMLST} e possibilitar análises locais e privadas com base numa nomenclatura alélica comum. O Chewie-NS combina várias tecnologias através de \textit{containerization}, sendo constituído por dois componentes: um componente \textit{backend} e um component \textit{frontend}. O componente de \textit{backend} inclui as bases de dados para armazenar e gerir os dados dos utilizadores e dos esquemas \textit{wg/cgMLST}, bem como uma API que aceita e processa os pedidos dos utilizadores e disponibiliza os dados para o componente \textit{frontend}. A API permitiu desenvolver um conjunto de módulos para integração com o chewBBACA 3 de forma a disponibilizar funcionalidades para descarregar, carregar e sincronizar esquemas. A integração com o chewBBACA 3 simplifica a preparação de esquemas \textit{wg/cgMLST} para análises locais e escaláveis, a obtenção de novos alelos a partir dos esquemas remotos depositados no Chewie-NS, e a contribuição de novos alelos identificados localmente. O processo de sincronização mantém as nomenclaturas alélicas utilizadas pelos esquemas locais e remotos sincronizadas para assegurar que os resultados são comparáveis. Esta abordagem descentralizada contrasta com o modelo centralizado adoptado por outras plataformas \textit{web} para \textit{wg/cgMLST}, que requerem que os utilizadores carreguem os seus dados para a plataforma, levantando questões de escalabilidade e confidencialidade de dados. O componente \textit{frontend} efectua o \textit{render} do \textit{website} do Chewie-NS, disponibilizando a lista de esquemas disponíveis para descarregar e estatísticas relevantes sobre a composição dos esquemas e a diversidade genética. O \textit{website} também disponibiliza uma ligação para uma interface gráfica da API, permitindo que utilizadores com diferentes níveis de proficiência explorem a API de forma mais intuitiva e tenham acesso a dados detalhados sobre os esquemas e \textit{loci}.

Um novo esquema \textit{wgMLST} para \textit{Streptococcus pyogenes}, constituído por 3,044 loci, foi desenvolvido a partir de conjuntos de dados representativos da diversidade da espécie. Os \textit{loci} do esquema foram anotados funcionalmente com base em várias fontes. Um processo de curadoria por parte de uma especialista na espécie permitiu validar as anotações e refinar o esquema através da substituição ou remoção de \textit{loci} espúrios. As soluções criadas para resolver os problemas identificados durante o processo de refinamento podem ser integradas em fluxos de trabalho para melhorar a qualidade de esquemas e análises \textit{wg/cgMLST}. O esquema \textit{wgMLST} anotado melhora a resolução das análises comparativamente com métodos de tipagem mais clássicos, como PFGE e MLST de sete genes, apresentando desempenho semelhante a métodos baseados em \textit{SNPs}. A utilização de um esquema \textit{wgMLST}, em vez de um esquema \textit{cgMLST} mais estrito, permite análises \textit{cgMLST} escaláveis em que o conjunto de \textit{core loci} é ajustado com base no conjunto de dados em análise. O esquema proporciona alto poder discriminatório para caracterizar e distinguir estirpes num conjunto de dados representante da diversidade global de \textit{S. pyogenes}, bem como num contexto de surto para distinguir estirpes de linhagens recentes.

Em conclusão, os métodos e resultados apresentados nesta tese procuram melhorar as actuais abordagens para a caracterização de bactérias com base em sequenciação total do genoma. O chewBBACA 3 reduz os requisitos para análises \textit{wg/cgMLST} escaláveis e detalhadas. O web service Chewie-NS, mesmo não sendo tão completo do ponto de vista de funcionalidades comparativamente com outras plataformas \textit{wg/cgMLST}, simplifica o acesso a esquemas e procura minimizar problemas relacionados com escalabilidade e confidencialidade dos dados. O esquema \textit{wgMLST} para \textit{S. pyogenes} permite uma caracterização detalhada de estirpes a qualquer nível de resolução, e o seu desenvolvimento permitiu identificar problemas comuns e soluções para melhorar a qualidade de esquemas \textit{wg/cgMLST}.
