


Bacterial infections are a huge burden for public health systems globally, cUausing significant health and economic losses. The emergence of accurate and cost-efficient high-throughput DNA sequencing technologies, especially by enabling whole genome sequencing of bacterial genomes, has revolutionized the characterization of bacterial strains in applications such as surveillance and outbreak investigation. These technologies have been widely adopted by research and public health institutions, providing increased resolution in a wide range of applications to complement or replace more classical phenotypic and molecular assays. The wealth and complexity of the generated data demanded greater storage capacity and improved computational methods for data analysis to make sense of the data. These demands potentiated a tremendous growth in the field of bioinformatics, which has become an integral part in omics approaches such as genomics. In bacterial genomics, in particular, bioinformatics methods have become essential to characterize bacterial strains, providing higher resolution in areas such as infectious disease surveillance and outbreak investigation, as well as for the study of the structure and evolution of bacterial populations. Whole genome sequencing allows for a purely sequence-based approach for bacterial characterization, enabling targeted approaches to identify genes of interest or the analysis of the full gene diversity. The increased availability of complete or nearly complete bacterial genome assemblies allowed researchers to study the structure and variability of bacterial genomes and encouraged the development of approaches for high-resolution bacterial typing, such as gene-by-gene and SNP-based methods. Although both approaches have been extensively applied in comparative genomics, either separately or in combination, to study the diversity of bacterial populations, gene-by-gene methods such as whole- and core-genome multilocus sequence typing (wg/cgMLST) have been adopted more frequently by research and public health institutions. wg/cgMLST allows to create schemas to capture the loci diversity of species of interest. The schemas enable the characterization of bacterial strains and can be updated over time with new alleles to maximize applicability in the long term. This is a gradual process that is commonly performed by web platforms that centralize data analysis. The efficiency of data analyses in centralized systems may be sufficient for routine surveillance, but it raises scalability concerns when it is necessary to perform large-scale analyses in reduced time, especially with the increase in the number of genome assemblies that are publicly available. Furthermore, centralized systems require users to upload their data, which may not be possible for users or institutions working under stricter data privacy policies. In addition, most analyses are performed at the core genome level, targeting the set of loci that are present in most strains, but discarding less frequent loci that constitute a very significant part of gene diversity and may be determinant for relevant phenotypic characteristics, such as virulence and antimicrobial resistance. Thus, developing methods that allow for local and large-scale analyses, while also being able to integrate the diversity of accessory loci more accurately, can minimize scalability and data privacy concerns and considerably expand the resolution of wg/cgMLST. The present thesis presents methods that aim to improve the scalability, accuracy, and interoperability of wg/cgMLST.

The chewBBACA suite for wg/cgMLST served as the basis to explore and implement new methods for improved wg/cgMLST. The new methods were implemented into chewBBACA 3, which constitutes a complete reimplementation of its predecessor, chewBBACA 2. In contrast to chewBBACA 2, which evaluated the coding sequences predicted for each input genome separately, chewBBACA 3 identifies and stores the list of distinct coding sequences predicted from all input genomes to enable fast and non-redundant exact matching and classification at the DNA and protein levels based on sequence hash comparisons. In addition, chewBBACA 3 complements alignment-based allele identification with alignment-free methods, more specifically, minimizer-based clustering, allowing for faster and more accurate allele identification. Schema creation with chewBBACA 3 is up to 55-fold faster than with chewBBACA 2 and identifies up to 10\% more loci, allowing to capture more of the diversity of bacterial species. Allele calling with chewBBACA 3 is up to 20.3- and 51.9-fold faster than with chewBBACA 2 and a comparable method, respectively. Furthermore, chewBBACA 3 classifies more coding sequences and scales better than the other methods, allowing large-scale wg/cgMLST in reduced time with computational resources typically available on a laptop. chewBBACA 3 includes functionalities to generate interactive reports that allow for an intuitive and comprehensive evaluation of wg/cgMLST schemas and results. The reports provide results and functionalities to explore loci diversity and identify groups of similar strains, which are relevant for surveillance, outbreak detection, and population studies.

A Web service, Chewie-NS, was implemented to provide broad access to wg/cgMLST schemas and allow local and private analyses based on a common allelic nomenclature. Chewie-NS leverages containerization to combine various technologies into two main components: a backend and a frontend components. The backend component includes the databases used to store and manage user data and wg/cgMLST schemas, as well as an API that accepts and processes user requests and provides data for the frontend component. The API allowed to develop a set of modules for integration with chewBBACA 3 to provide functionalities for schema download, upload, and synchronization. The integration with chewBBACA 3 allows users to quickly set up a wg/cgMLST schema for local and scalable analysis, retrieve novel alleles added to the remote schemas in Chewie-NS, and contribute with novel alleles identified locally only if desired. The synchronization process maintains the allelic nomenclatures used by local and remote schemas synchronized to ensure the comparability of the results. This decentralized approach contrasts with the centralized model adopted by other well-established web platforms for wg/cgMLST, which require users to upload their data to the platform, raising scalability and data privacy concerns. The frontend component renders the Chewie-NS website, providing easy access to the list of available schemas for download and relevant statistics about schema composition and loci diversity. In addition, the website links to a graphical interface for the API, which allows users of any level of expertise to explore the API more intuitively and retrieve detailed schema and loci data.

A novel wgMLST schema for \textit{Streptococcus pyogenes}, comprising 3,044 loci, was developed based on datasets representative of the species diversity. The loci in the schema were annotated by retrieving functional annotation data from several sources. A careful curation process by a domain expert allowed to validate the loci annotations and refine the schema by substituting or removing spurious loci. The solutions created to resolve the issues identified during the curation process can be integrated into workflows to improve the quality of wg/cgMLST schemas and analyses. The annotated wgMLST schema provides increased resolution compared to more classical typing methods, such as PFGE and seven-gene MLST, and displays performance comparable to SNP-based methods. Using a wgMLST schema, instead of a hard-defined cgMLST schema, enables scalable cgMSLT analysis where the set of core loci is adjusted based on the dataset under analysis. The schema provided high discriminatory power to characterize and distinguish strains in a dataset representing the global diversity of \textit{S. pyogenes}, as well as in outbreak context to distinguish strains from recently emerged lineages.

In conclusion, the methods and results presented in this thesis seek to improve current approaches for bacterial characterization based on whole genome sequencing. chewBBACA 3 lowers the barrier for scalable and comprehensive wg/cgMLST. Chewie-NS, while not as feature-complete as other well-established platforms, provides easy access to schemas and aims to minimize scalability and data privacy concerns. The wgMLST schema for \textit{S. pyogenes} allows detailed strain characterization at any resolution level, and its development enabled the identification of key issues and solutions to improve the quality of wg/cgMLST schemas.
