
(Include something about the burden of bacterial infections)

The emergence of accurate and cost-efficient high-throughput DNA sequencing technologies has revolutionized biological sciences. These technologies have been widely adopted by research and public health institutions, providing increased resolution in a wide range of applications to complement or replace more classical phenotypic and molecular assays. The wealth and complexity of the generated data demanded greater storage capacity and improved computational methods for data analysis to make sense of the data. These demands potentiated a tremendous growth in the field of bioinformatics, which has become an integral part in omics approaches such as genomics. In bacterial genomics, in particular, bioinformatics methods have become essential to characterize bacterial strains, providing higher resolution in areas such as infectious disease surveillance and outbreak investigation, as well as for the study of the structure and evolution of bacterial populations. Whole genome sequencing allows for a purely sequence-based approach for bacterial characterization, enabling targeted approaches to identify genes of interest or the analysis of the full gene diversity. The increased availability of complete or nearly complete bacterial genome assemblies allowed researchers to study the structure and variability of bacterial genomes and encouraged the development of approaches for high-resolution bacterial typing, such as gene-by-gene and SNP-based methods. Although both approaches have been extensively applied in comparative genomics, either separately or in combination, to study the diversity of bacterial populations, gene-by-gene methods such as whole- and core-genome multilocus sequence typing have been adopted more frequently by research and public health institutions for the surveillance and outbreak investigation of bacterial pathogens. These methods allow to create schemas to capture the loci diversity of species of interest, and which are expanded overtime with alleles identified in bacterial strains. This is a gradual process that is commonly performed by web platforms that centralize data analysis. The efficiency of data analyses in centralized systems may be sufficient for routine surveillance, but it raises scalability concerns when it is necessary to perform large-scale analyses in reduced time, especially with the increase in the number of genome assemblies that are publicly available. Furthermore, centralized systems require users to upload their data, which may not be possible for users or institutions working under stricter data privacy policies. In addition, most analyses are performed at the core genome level, targeting the set of loci that are present in most strains, but discarding less frequent loci that constitute a very significant part of gene diversity and may be determinant for relevant phenotypic characteristics, such as virulence and antimicrobial resistance. Thus, developing methods that allow for local and large-scale analyses, while also being able to integrate the diversity of accessory loci more accurately, can minimize scalability and data privacy concerns and considerably expand the resolution of whole- and core-genome multilocus sequence typing. The present thesis presents methods that aim to improve the scalability, accuracy, and interoperability of whole- and core-genome multilocus sequence typing.



