This thesis describes the development and application of improved bioinformatics methods for bacterial typing based on \ac{WGS}, more specifically \ac{GbG} approaches such as \ac{wg/cgMLST}. \ac{wg/cgMLST} is widely used for the characterization of bacterial strains by research and public health institutions for surveillance, outbreak detection, and for the study of the population diversity of bacterial species. Therefore, improving \ac{wg/cgMLST} methodologies is of high relevance and has a direct impact on research and public health, since any improvement can easily be adopted by the scientific community and public health authorities who already apply these methodologies. The thesis is organized into five chapters.

\textbf{\autoref{ch:introduction}} is a general introduction to bacterial typing methods, starting with a brief explanation of some of the classical biochemical and molecular methods used in the laboratory and extending this to transition to the \ac{WGS}-based approaches made possible by the DNA sequencing revolution. The main WGS-based approaches for bacterial typing are explained, with a greater focus on \ac{GbG} approaches, such as \ac{wg/cgMLST}, which are explained more thoroughly.

The general introduction is followed by the body of the thesis, composed of three selected manuscripts that present methods and results that aim to improve the efficiency, accuracy, and interoperability of \ac{wg/cgMLST}. Each manuscript has its own dedicated chapter. There is a high degree of intersection between the subjects of each chapter, and the work of each chapter contributes in part or is instrumental to advance the work presented in the other chapters.

\textbf{\autoref{ch:paper1}} describes the implementation and evaluation of a bioinformatics tool for \ac{wg/cgMLST}, chewBBACA 3. chewBBACA 3 is a reimplementation of the first published version of chewBBACA and was designed to be a complete solution for \ac{wg/cgMLST}, offering improved and new functionalities that considerably expand chewBBACA's capabilities. It enables fast and efficient schema creation from multiple sources and allele calling with modest computational resources to meet current and future data processing demands. The speed and scalability of the allele calling opens up the possibility of large-scale \ac{wg/cgMLST} for more users who want to study the population diversity of a species of interest based on large datasets or in surveillance and outbreak scenarios where a timely analysis is crucial. To take advantage of the extensive schema and results data produced, chewBBACA 3 includes modules for a comprehensive analysis of the schemas and allele calling results. These modules produce interactive and easily shareable reports based on an upgraded version of the \ac{UI} developed for the \ac{WS} presented in \textbf{\autoref{ch:paper2}}, allowing users of any level of expertise to more easily explore the results and make informed decisions.

\textbf{\autoref{ch:paper2}} presents \ac{Chewie-NS}, a \ac{NS} that stores \ac{wg/cgMLST} schemas and allows for local and private \ac{wg/cgMLST} analysis based on a common allelic nomenclature. The clear and simple interface of \ac{Chewie-NS} allows users to easily find, explore loci diversity, and download schemas for species of interest. For users who are more tech-savvy, the \ac{API} provides access to all schema data, either through Swagger \ac{UI} or programmatically. The schemas downloaded from \ac{Chewie-NS} can be used to perform local and private analysis, minimizing scalability issues if data analysis was centralized in a \ac{WS} and allowing users operating under stricter data privacy policies to still take advantage of the data and functionalities provided by \ac{Chewie-NS}. This contrasts with other \ac{wg/cgMLST} platforms, which typically centralize data analysis, requiring users to submit their data and establishing queues to process data submissions based on available computational resources. The local schemas downloaded from \ac{Chewie-NS} can be synchronized with the remote versions to retrieve novel alleles added to the remote schemas and contribute novel alleles identified locally if desired. The synchronization process allows users to update both schema versions and the allelic nomenclature, enabling results comparison even if data analysis is not centralized. To provide functionalities for users to use the schemas deposited in \ac{Chewie-NS}, a set of modules was developed for chewBBACA 3, presented in \textbf{\autoref{ch:paper1}}, allowing users to upload, download and synchronize the schemas. The integration with chewBBACA 3 provides access to the schemas deposited in \ac{Chewie-NS}, allowing users to perform local allele calling and compare their results based on a common allelic nomenclature, while also allowing users to take advantage of chewBBACA's powerful analytic capabilities.

\textbf{\autoref{ch:paper3}} describes the creation and evaluation of an annotated \ac{wgMLST} schema for \textit{S. pyogenes}, a major human pathogen with high genetic diversity. The target loci for the \ac{wgMLST} schema were defined based on a dataset of high-quality complete genomes and the schema was populated by allele calling with multiple datasets representing the known diversity of \textit{S. pyogenes}. This schema was refined based on automatic annotations and the suggestions of a domain expert. The refined \ac{wgMLST} schema, comprising 3,044 loci, allows for high-resolution typing of \textit{S. pyogenes} for the study of diverse datasets and in an outbreak context, showing performance comparable to SNP-based methods. The development of the \ac{wgMLST} schema benefited from the novel functionalities that were being implemented in chewBBACA 3 concomitantly, presented in \textbf{\autoref{ch:paper1}}. The schema refining process also revealed key limitations and challenges of scaling current \ac{GbG} approaches, often based only on \ac{cgMLST}, to \ac{wgMLST}, and provided crucial information to guide the development of some functionalities included in chewBBACA 3. The \ac{wgMLST} schema was deposited in \ac{Chewie-NS}, presented in \textbf{\autoref{ch:paper2}}.

\textbf{Chapter 5} corresponds to the general discussion. In this chapter, the results presented in each chapter of the body of the thesis are summarized and discussed in terms of their advantages, disadvantages, and the potential to be further improved.
